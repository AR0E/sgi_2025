\chapter{Day 3}

\section{Geometric data}

It takes much effort to represent shapes in a virtual environment. Since
we only have finite memory and processing power, it is wise to implement
different data types for representing different constructs.

\spa

For example, when scanning real objects we're not really able to infer the
contextual information regarding certain features; therefore, we end up with
simple points in 3D space, which are clouds, but if you want to represent something
akin to a curve or do precision engineering work, you might want a mathematically
parametrized surface, since those don't suffer from finite resolution and can
be modified indefinitely.

\begin{table*}[h] \centering
%\ra{1.3}
\begin{small}
\begin{tabular}{@{}lll@{}}\toprule
\textbf{Data Type} & \textbf{Dimen.} & \textbf{Note} \\ \midrule
\textbf{Point clouds} & 0/3 & 52  \\ \hdashline
\textbf{Polylines} & 1/3 & 88  \\ \hdashline
\textbf{Splines} & 1/3 & 71  \\ \hdashline
\textbf{Implicit Surfaces} & 2/3 & 60  \\ \hdashline
\textbf{Parametric surfaces} & 2/3 & 55\\ \hdashline
\bottomrule
\end{tabular}
\end{small}
\caption{Data types and their corresponding co-dimensionality as well
as use in embedded spaces. Point clouds are, for example, 0-dimensional
entities that can be embedded in 3D space.}
\end{table*}

\subsection{Point Clouds}

Still don't know much about them besides just being an aggregate cluster of
coordinates.

\subsection{Splines / Curves}

Discretize a curve $\gamma(t)$ into a set of finite points
by using splines, which are simple parametrizations that obey
certain differentiability rules; see 
\href{https://www.youtube.com/watch?v=aVwxzDHniEw}{Freya Holmer's talk
on Bezier's} and on the 
\href{https://www.youtube.com/watch?v=jvPPXbo87ds}{the continuity of
splines}.

\subsection{Implicit Shape Representations}

\emph{Implicit Shape Representations} are of the form 
$f(\bf{x})=0$; this differs from \emph{parametric} forms
of characterizing $g(u,v)$. One of the major advantages is the
ease of computing intersections \cite{ray1} by means of a
bounding hierarchy and a simple algorithm to check
wheter a ray $\bf{r}(t)=\bf{a}+t\bf{b}$ crosses
it or not.

\spa

Further, these implicit surfaces can also have constructive
geometric operations applied to them very easily. An example is
a basic \texttt{AND/OR} algorithm by taking the $\min$ and
$\max$ of the two functions, or the deformation of
\emph{meta balls}, which is a set of sum of gaussians.

\subsection{Signed Distance Functions} \label{SDFs}

Following \cite{wave1} and \cite{SDF1}, a \emph{SDF} is a function that
determines the \emph{distance} from an arbitrary point in space to the
object in question. In 2D space, for example, imagine we had
a shape $\mathcal{S}$, and we pick some arbitrary point 
$\bf{x}$ in the space around the shape, $\mathcal{M}$. We want
a point $\bf{p} \in \mathcal{S}$ that \emph{minimizes} the distance
function between the point in the shape and the point in space:

\begin{align*}
    \min \text{distance}(\bf{x}, \cal{S}) = d(\bf{x,p})
\end{align*}

To produce the function 
$\text{SDF}(\bf{x,p}) : \RR^n\rightarrow\RR$ that
takes a point in $n-$dimensional space and spits out a distance scalar,
we add an aditonal constraint to this distance function:

\begin{align*}
\text{SDF}(\bf{x,p}) =
\begin{cases}
\ \ d(\bf{x,p}), \quad \text{if }\bf{x} \notin \mathcal{S} \ \text{(exterior)}
 \\
-d(\bf{x,p}), \quad \text{if }\bf{x} \in \mathcal{S} \ \text{(interior)}
\end{cases}
\end{align*}

That is, if you pick a point \emph{inside} the shape, it'll also
register the distance, but it'll be negative. The points \emph{on}
the shape have 0 distance, and so are not signed. There are also
simpler but less used \emph{indicator functions} that return 1 outside
and $-1$ inside.

\spa

One can have \emph{exact} and \emph{approximate} SDFs. An 
\emph{exact} SDF obeys the \emph{eikonal equation} $|\grad f| = 1$, as is
explained by \cite{wave1} in equation 6. The idea being very simple:
imagine we have a tank of water, and we twist a metal wire in the shape 
we want, and then we \emph{splash} that shape in the water.

\spa

The outward and inward propagating waves out of this splash produce
exactly the level sets of distance of our shape; so, if we compute how
a wave propagates in that way, and grab the level sets or the wave fronts
of the wave over time, we get the SDF. We can also prove the eikonal
condition using some functional inequalities \cite{wave2}.

\spa

Now to render and use these things we want to do some kind of algorithm
that checks the closest point to the shape at any moment. This is pretty
trivial to do with \emph{sphere tracing}. As the name implies, you pick
some point $\bf{x}$, draw a \emph{circle} $C$ around said point,
and start expanding its radius. At some point it'll touch some bit of
the shape, and that's the nearest bit to $\bf{x}$ \cite{SDF2}.
Easy right?

\spa

To construct more sophisticated SDFs we wish to use methods
to join basic, exact SDFs into other, more complicated shapes.
The basic procedure is to apply various \emph{boolean operations}
to the SDF, union, intersections and such, that produce new SDFs.
I'm not sure how this procedure is exactly done yet.

\subsection{Conversions between geometric representations}

Some things are impossible to do in certain representations. Animations
are near-inconceivable using parametric surfaces, but trivial in meshes.
How does one convert between these types of objects, without losing too
much information?
