\chapter{Day 2}

\section{Mesh Parametrization}

\begin{definition}[UV Maps]

The idea is we want to "squish" or "flatten" our 3D mesh into a snug
2D projection, an image. This produces a \emph{\textbf{UV map}}, and
the procedure in abstract is to assign a subset of 3D space 
$V$ contained by edges to a domain in 2D space $\Omega$:

\begin{align*}
    U: V\subset \RR^3 \rightarrow \Omega \subset \RR^2
\end{align*}
    
\end{definition}

If the edge is a closed object, like a cube, obviously if you try to squish
it, some bits of the cube will be on top of each other, so a no-go to
a UV map. What can be done is cheating; you cut some bits of the original
3D object first so that it becomes "flattable".

\spa

This procedure of incision is sucessful once we reach a
\emph{disk topology}, containing a single boundary loop. Some meshes
require that we make multiple seams and separate multiple sections into
little "UV islands"; this can be done for semantic or mathematical
purposes. Our objective in parametrization is to minimize
\emph{distortion} and the number of cuts.

\subsection{Springs algorithm (Tutte Embedding)}

A basic mesh parametrization algorithm treats the entire mesh as a
set of springs with different elastic coefficients; we want to
"relax" the whole mesh by allowing the springs to stretch until they're
flat and constrained to some boundary. This is known as a
\emph{\textbf{force-directing algorithm}} for making planar graphs.
A \href{https://www.youtube.com/watch?v=WWm-g2nLHds&list=PLubYOWSl9mIvtnRjCCHP3wqNETTHYjQex}{good playlist.}

\spa

This is the same as trying to minimize the potential energy $U$
of the entire system of springs.

\begin{align*}
\min_{U} \sum_{\{i,j\} \in \bf{E}} w_{ij}\bf{||u_j-u_i||}^2
\approx \min_{\text{Pot. energy}} \sum_{\text{vertices}} k\bf{x}^2
\end{align*}

We can numerically solve the relaxation of springs by looking at a basic
$2k\bf{x}=0$ idea:

\begin{align*}
\frac{\partial U}{\partial\bf{u}_i}=\sum_{\{i\,j\}} 2w_{ij} 
(\bf{u}_i-\bf{u}_j)=0 \approx 2k\bf{x}=0
\end{align*}

The spring potential $k$ represented by the stiffness of each edge
$w_{ij}$ can be adjusted to not be a constant, but rather depend on
something. Following \cite{param1stanford}, we have:

\begin{itemize}
    \item \textbf{Uniform weights:} The stiffness of every spring
    is equal. This relaxes them all into equilateral triangles.

    \item \textbf{Harmonic weights:} Dependant on adjacent angles
    between faces to imitate a conformal transformation. According
    to \emph{Lemma 1} in \cite{minimal1}, it comes about from
    minimizing the total discrete \emph{Dirichlet Energy}.

    \item \textbf{Mean-value weights:} Using barycentric coordinates,
    explained below, we can produce weights that are always positive
    and precise in the linear sense.
\end{itemize}

Following \cite{param2sheffer}

\spa

\subsection{Distortion}

The amount of distortion a mesh has is measured by the Jacobian, which
tells you the total infinitesimal variation in a volume element across
all directions.

\begin{align*}
    J_f = U\Sigma V^T = U
\begin{pmatrix}
\sigma_1 & 0\\
 0 & \sigma_2\\
 0 & 0
\end{pmatrix}
V^T
\end{align*}

We can split this into three cases.

\begin{itemize}
    \item \textbf{Area preserving (Equiareal):} 
    If $\text{det}(J_A^2)=\text{det}(J_B^2)$, this implies
    $\sigma_1 \sigma_2 = 1$.
    
    \item \textbf{Conformal (Equiangular):} If there exists a scalar function
    $\phi(u,v)$ that perfectly offsets the variation of the original
    area, such that $J_A^2 = \phi J_B^2$, this is conformal, and it
    implies $\sigma_1=\sigma_2$.
    
    \item \textbf{Isometric:} If the transformation is both
    conformal and equiareal, then it is isometric, and therefore implies
    $J^2_A = J^2_B$, which implies $\sigma_1=\sigma_2=1$.
\end{itemize}

\href{https://github.com/alecjacobson/geometry-processing-parameterization}{Jaconson's course}

\subsection{Barycentric Interpolation}

As per the original floater article \cite{floater1}, as well
as his more explanatory article \cite{floater2}, the basic idea is that
the coordinates \emph{inside} of a triangle $T$ composed of vertices
$\bf{v_1, v_2}$ and $\bf{v_3}$ are linear superpositions of the
vertices scaled by a ratio of the area of the triangles formed by each
vertex and the total triangle area:


\begin{align*}
\bf{x} &= \lambda_1 \bf{v_1} + \lambda_2 \bf{v_2} 
+ \lambda_3 \bf{v_3} \\
\\
\lambda_i&=
\begin{pmatrix}
\text{Area of} \\
\text{triangle made of} \\
\text{base edge } \bf{v}_i
\end{pmatrix}
\bigg/
\begin{pmatrix}
\text{Total area} \\
\text{of triangle}
\end{pmatrix}
\end{align*}

The construction can actually be generalized to any convex polygon
(altough not at all polygons will produce bijective maps \cite{barycentric1})
but we'll stick to triangles. The original motivation of this
system was to mimic \emph{harmonic maps} by piecewise linear maps, that
is, allow a conformal transformation to be more easily carried out.

\spa

Suppose we had a harmonic function $\Delta f = 0$. If we wish to
approximate $f$ by regarding a boundary condition 
$u(\partial \Omega) = u_0$. We want to produce some triangulation
$\mathcal{T}$ whose values of interior edge lengths and angles
look like a conformal map. This then produces a system of equations
that is exactly that of mean value, or barycentric coordinates.

\spa

For mesh parametrization, we want to follow
the procedure outlined in section 5.4 of \cite{floater2}.
Suppose we have a mesh $\bf{M}(E,V, F)$, where $E$ are its edges,
$V$ its vertices and $F$ its faces. Then the \emph{boundary of the mesh}
$\partial\bf{M}$ produces a \emph{boundary polygon}.

\spa

Our goal is to, as per the Tutte embedding, map the boundary polygon
into another convex polygon fixed in the UV plane, and then make all
other triangles of the mesh be convex combinations of their neighbors
in such a fashion as to fill the UV plane.

\spa

Mean coordinates can also be extended to non-convex polygons.

\section{Texture mapping and optimization}

\subsection{Mapping}

Supposing we express a mesh $\bf{M(u,v)}$ in these UV coordinates,
we can store image data in the UV plane and then map this back to the
original mesh to get colors on our mesh.

\spa

The basic method is a \emph{\textbf{triangle soup}} mapping, which
obliterates connectivity information in return for a completely
isometric mapping to the UV space.

\subsection{Optimization}