\chapter{Exercises from SGI}

\section{Exercises 1-13}
\subsection{Finding boundary edges}

Here's the idea; if we imagine that each face is stored
as a row $[\bf{v_1, v_2, v_3}]$ in the face matrix
$\bf{F}$, and we want to find boundary edges, we must
find pairs $\bf{(v_1, v_2)}$, not necessarily in order,
that don't repeat in any row.



Since we're indexing $\bf{v}_i$ as numbers in each row,
we then are tasked to find all sets of non-repeating tuples
across a matrix $n \times 3$ $\bf{F}$.

\begin{algorithm}
\caption{Finding boundary edges of a mesh by substring counting}
\begin{algorithmic}[1]
\Procedure{BoundEdges1}{$\bf{F}$}

\State $\bf{F} \gets \RR^{n \times 3}$
\Comment{Initialize face matrix}
    
    \State $\bf{A} \gets \bf{F}[i,:]$
    \Comment{Gets all elements from row $i$}
    \For{$i = 0, \cdots, n$}
    \For{$j = 0, \cdots, n$}
    \For{$k = 0, \cdots, n$}
    \EndFor
    
    
\State \Return UPairs
\State $\bf{BE} \gets \text{UPairs}$
\State \textbf{print(BE)}
\EndProcedure
\end{algorithmic}
\end{algorithm}

This algorithm runs in $\mathcal{O}(k^2n)$ time, where
$n$ is the number of rows in $\bf{F}$, and $k$ is the
number of columns. Therefore it becomes a $O(n)$ algorithm
in regards to number of faces.


\subsection{Finding boundary length}

This one is easy. Once we have all boundary edges,
we can take the distance between the vertices of each edge,
and then repeat the process for all edges.

\begin{algorithm}
\caption{Finding total boundary perimeter}
\begin{algorithmic}[1]
\Procedure{BoundLen}{$\bf{F}$}

\EndProcedure
\end{algorithmic}
\end{algorithm}

\subsection{Per-face normals}

Every face is identified with a collection
$\bf{(v_1,v_2,v_3)}$ of vertices, where their
orientation is counter-clockwise; by producing
the vectors $\bf{(v_2-v_1)=v_a}$ and $\bf{(v_3-v_1)=v_b}$
we can produce the normal vector $\bf{v_a \times v_b}$.
However, we must then normalize it by grabbing the
magnitude of $\bf{|v_a \times v_b|}$ and dividing
it to our acquired vector.

\begin{algorithm}
\caption{Vector normalization algorithm}
\begin{algorithmic}[1]
\Procedure{VNorm}{$\bf{v}$}
\State $\bf{v} \gets \RR^n$ \Comment{Load vector}
\State $f \gets 0$
    \For{$i = 0, \cdots, n$}
    \State $f \gets f + \bf{v}[i]^2$
    \State $f \gets \sqrt{f}$
    \State $\bf{v}[i] \gets \bf{v}[i]/f$
    \State \Return $\bf{v}[i]$ \Comment{Normalized vector}
    \EndFor
\EndProcedure
\end{algorithmic}
\end{algorithm}

\begin{algorithm}
\caption{Finding the normal of a face}
\begin{algorithmic}[1]
\Procedure{FaceNorm}{$\bf{F}$}
\State $\bf{F} \gets \RR^{n \times 3}$
\State $\bf{V} \gets \RR^{n \times 3}$
\For{$i = 0, \cdots, n-1$}
    \State $\bf{A} \gets \bf{F}[i, :]$ 
    \Comment{Acquire a row of $\bf{F}$}
    \State $\bf{A[V[0, j]]}$
    \State $\bf{v_a = A[V[0,1]] - A[V[0,0]]} $
    \State $\bf{v_b = A[V[0,2]] - A[V[0,0]]} $
    \State $\bf{n} \gets \text{Cross}(\bf{v_a,v_b})$
    \State $\bf{n} \gets \text{VNorm}(\bf{n})$ \Comment{Normalize}
\EndFor
\EndProcedure
\end{algorithmic}
\end{algorithm}

This procedures occurs in $\mathcal{O}(n)$ time.

\subsection{Per-vertex normals}

As discussed in section \ref{per-vertex-normals}, to compute
the  per-vertex normal we need to first sample all the faces
around a given vertex, compute all of their normal vectors,
compute the associated \emph{weight factor}, and then
sum them all together. After that we normalize this final
vector.

\begin{algorithm}
\caption{Per-vertex normal}
\begin{algorithmic}[1]
\Procedure{PVNorm}{$\bf{F}$}

\EndProcedure
\end{algorithmic}
\end{algorithm}

\subsection{Flipped normals}

We just take the vector array and negate all its
three components $\bf{n \rightarrow -n}$.
Not really sure if there is a sophisticated idea behind
this one that's better.

\begin{algorithm}
\caption{Vector flipping algorithm}
\begin{algorithmic}[1]
\Procedure{VNorm}{$\bf{v}$}
\State $\bf{v} \gets \RR^n$ \Comment{Load vector}
    \For{$i = 0, \cdots, n$}
    \State $\bf{v}[i] \gets -\bf{v}[i]$ \Comment{Negate components}
    \State \Return $\bf{v}[i]$ \Comment{Flipped vector}
    \EndFor  
\EndProcedure
\end{algorithmic}
\end{algorithm}

\subsection{Tangents}

According to \cite{tangent1, tangent2, foundationsofgameengine1}, after
acquiring $\bf{(v_2-v_1)=v_a}$ and $\bf{(v_3-v_1)=v_b}$,
we can acquire the \emph{tangent vector} $\bf{t}$
and \emph{bitangent vector} $\bf{b}$, both compose a
basis on the face, by solving a linear system:

\begin{align*}
    [\bf{v_a} \ \bf{v_b}] = [\bf{t  \ \ b}]
    \begin{pmatrix}
 x_1 & x_2 \\
 y_1  & y_2
\end{pmatrix}
\end{align*}

Where $\{\bf{x, y}\}$
This can be done by inverting the right-hand side matrix and multiplying:

\begin{align*}
    [\bf{t  \ \ b}] = (x_1 y_2 - x_2 y_1)^{-1} [\bf{v_a \ \ v_b}]
\begin{pmatrix}
 y_2 & -x_2 \\
 -y_1  & x_1
\end{pmatrix}
\end{align*}



\begin{algorithm}
\caption{Finding tangent vectors}
\begin{algorithmic}[1]
\Procedure{BoundLen}{$\bf{F}$}

\EndProcedure
\end{algorithmic}
\end{algorithm}

\subsection{Angle Defect} \label{angle-defect-algorithm}

As per \cite{discreteexterior1} and \ref{curvature}, the
total discrete angle defect in a mesh is given by:

\begin{align*}
    \theta_{\text{def}}(\bf{v}) = 2\pi - \sum_{f \in F}
    \text{Interior Angle}(\bf{v})
\end{align*}

Therefore, as in the per-vertex normal case, we must pick a vertex 
$\bf{v}$, loop around the faces $F$ around the vertex, and at each one
compute the total interior angle $\theta_i$ based around $\bf{v}$,
and then assign the value of the formula for that vertex.
We then do this for all vertices in the mesh.

\spa

Now to compute the interior angles, we need to produce
two vectors $\bf{(v_a,v_b)}$ whose origin is the
vertex $\bf{v}$, and then perform a numerically stable
routine as per \cite{float3}.


\begin{algorithm}
\caption{Finding angle defect $\theta$ across entire mesh}
\begin{algorithmic}[1]
\Procedure{AngDef}{$\bf{F, V}$}


\EndProcedure
\end{algorithmic}
\end{algorithm}

\subsection{Four corners}

\begin{algorithm}
\caption{Finding total boundary perimeter}
\begin{algorithmic}[1]
\Procedure{BoundLen}{$\bf{F}$}

\EndProcedure
\end{algorithmic}
\end{algorithm}

\subsection{My diagonals}

\begin{algorithm}
\caption{Finding total boundary perimeter}
\begin{algorithmic}[1]
\Procedure{BoundLen}{$\bf{F}$}

\EndProcedure
\end{algorithmic}
\end{algorithm}

\subsection{Sparse matrix with a triangular nonzero pattern}

\begin{algorithm}
\caption{Finding total boundary perimeter}
\begin{algorithmic}[1]
\Procedure{BoundLen}{$\bf{F}$}

\EndProcedure
\end{algorithmic}
\end{algorithm}

\subsection{Upsample}

\begin{algorithm}
\caption{Finding total boundary perimeter}
\begin{algorithmic}[1]
\Procedure{BoundLen}{$\bf{F}$}

\EndProcedure
\end{algorithmic}
\end{algorithm}

\subsection{Grad}

\begin{algorithm}
\caption{Evaluating the gradient}
\begin{algorithmic}[1]
\Procedure{Grad}{$\bf{V,F}$}

\EndProcedure
\end{algorithmic}
\end{algorithm}

\subsection{Critical Points}

\begin{algorithm}
\caption{Finding critical points in mesh}
\begin{algorithmic}[1]
\Procedure{BoundLen}{$\bf{F}$}

\EndProcedure
\end{algorithmic}
\end{algorithm}

\subsection{Cotangent Laplacian from grad}

\begin{algorithm}
\caption{Cotangent Laplacian from grad}
\begin{algorithmic}[1]
\Procedure{BoundLen}{$\bf{F}$}

\EndProcedure
\end{algorithmic}
\end{algorithm}

\subsection{Cotangent Laplacian from entries}


\begin{algorithm}
\caption{Cotangent Laplacian from entries}
\begin{algorithmic}[1]
\Procedure{BoundLen}{$\bf{F}$}

\EndProcedure
\end{algorithmic}
\end{algorithm}